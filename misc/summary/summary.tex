\documentclass[a4paper]{article}
\usepackage[hungarian,english]{babel}
\usepackage[utf8]{inputenc}

\begin{document}
\begin{titlepage}
\begin{center}

\LARGE{Validation and Improvement of the SMPI simulation framework for
MPI applications}\\[0.8cm]
\emph{\large{Project Summary}}\\[0.5cm]
\rule{\linewidth}{0.5mm}\\[1.5cm]
\emph{\normalsize{Attila Döme Lehóczky}}\\[0.5cm]
\textsc{\normalsize{Cranfield University, School of Engineering}}

\vfill

{\large \today}

\end{center}
\end{titlepage}

\clearpage
\setcounter{page}{1}
\paragraph{Cross-platform testing and validation framework}
SMPI aims at predicting the performance of (potentially
computation-heavy) MPI applications. It is important to note that MPI
has multiple notable implementations, which can have differences in
performance. Validation of predictions for the different
implementations is a tedious task, as a lot of the code has to be
changed to do it. One of the research directions is to create a
testing and validation framework which we could use to create tests
with the ability of seamlessly - with as little intervention from the
user as possible - switching between the MPI implementations. The
first aim is to do so with the MPI implementations OpenMPI and MPICH.
\paragraph{STAR-MPI approach}
Self-Tuned Adaptive Routines for MPI Collective Operations (STAR-MPI)
is a sadly discontinued project, but one which's ideas could be
utilized in SMPI. It is a set of MPI collective communication routines
that are capable of dynamically adapting to system architecture and
application workload. The main idea lays in a technique called "delayed
finalization of MPI collective communication routines" (DF). For each
operation, STAR-MPI maintains a set of communication algorithms. The
aim is to postpone the decision of which algorithm to use until after
the platform and/or the application is known. This technique bears
the potential of platform-specific or application-specific
optimalization of an MPI application.\\
A development idea for SMPI is to apply the same technique there: a
set of potentially choosable algorithms could be implemented alongside
a set of selector mechanisms. By using the STAR-MPI approach,
extensive testing could be conducted on SMPI with different tuning
parameters. The results of these tests could be used to suggest
better parameters for the actual MPI implementations, in order to help
improve their performance.
\end{document}
