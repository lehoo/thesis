%% ----------------------------------------------------------------
%% Thesis.tex -- MAIN FILE (the one that you compile with LaTeX)
%% ----------------------------------------------------------------

% Set up the document
\documentclass[a4paper, 11pt, oneside]{Thesis}  % Use the "Thesis" style, based on the ECS Thesis style by Steve Gunn
\graphicspath{{Figures/}}  % Location of the graphics files (set up for graphics to be in PDF format)

% Include any extra LaTeX packages required
\usepackage[square, numbers, comma, sort&compress]{natbib}  % Use the "Natbib" style for the references in the Bibliography
\usepackage[english, hungarian]{babel}
\usepackage[utf8]{inputenc}
\usepackage[T1]{fontenc}
\usepackage{verbatim}  % Needed for the "comment" environment to make LaTeX comments
\usepackage{vector}  % Allows "\bvec{}" and "\buvec{}" for
                     % "blackboard" style bold vectors in maths
\usepackage{xr}
\usepackage{hyperref}
\usepackage[table]{xcolor}
\usepackage{multirow}
\usepackage{array,booktabs}
\hypersetup{urlcolor=blue, colorlinks=true}  % Colours hyperlinks in blue, but this can be distracting if there are many links.

%%Renewing some commands so they don't appear in Hungarian
%%names gotten from: %%/usr/share/texmf-texlive/tex/generic/babel/magyar.ldf
%%modified line 111 there to change how the figure name is displayed

\addto\captionshungarian{%
  \renewcommand{\listfigurename}%
    {List of Figures}%
}

\addto\captionshungarian{%
  \renewcommand{\listtablename}%
    {List of Tables}%
}

\addto\captionshungarian{%
  \renewcommand{\bibname}%
    {Bibliography}%
}

\addto\captionshungarian{%
  \renewcommand{\contentsname}%
    {Contents}%
}

\addto\captionshungarian{%
  \renewcommand{\figurename}%
    {Figure}%
}

\addto\captionshungarian{%
  \renewcommand{\chaptername}%
    {Chapter}%
}

\addto\captionshungarian{%
  \renewcommand{\tablename}%
    {Table}%
}

\newcommand{\methodname}[1] {
\\[0.2cm]
\ttfamily #1\normalfont
}

%% ----------------------------------------------------------------
\begin{document}
\frontmatter      % Begin Roman style (i, ii, iii, iv...) page numbering

% Set up the Title Page
\title  {A testing framework and workflow for validation and
  improvement of the SMPI simulation framework for MPI applications}
\authors  {\texorpdfstring
            {\href{your web site or email address}{Attila Döme Lehóczky}}
            {Attila Döme Lehóczky}
            }
\addresses  {\groupname\\\deptname\\\univname}  % Do not change this here, instead these must be set in the "Thesis.cls" file, please look through it instead
\date       {\today}
\subject    {}
\keywords   {}

\maketitle
%% ----------------------------------------------------------------

\setstretch{1.3}  % It is better to have smaller font and larger line spacing than the other way round

% Define the page headers using the FancyHdr package and set up for one-sided printing
\fancyhead{}  % Clears all page headers and footers
\rhead{\thepage}  % Sets the right side header to show the page number
\lhead{}  % Clears the left side page header

%% ----------------------------------------------------------------
% The "Funny Quote Page"
\pagestyle{empty}  % No headers or footers for the following pages

\null\vfill
% Now comes the "Funny Quote", written in italics
\textit{``Groovy!''}

\begin{flushright}
%If the quote is taken from someone, their name goes here
\end{flushright}

\vfill\vfill\vfill\vfill\vfill\vfill\null
\clearpage  % Funny Quote page ended, start a new page
%% ----------------------------------------------------------------

% The Abstract Page
\addtotoc{Abstract}  % Add the "Abstract" page entry to the Contents
\abstract{
\addtocontents{toc}{\vspace{1em}}  % Add a gap in the Contents, for aesthetics

SimGrid is a generic simulation framework, which can be used to
simulate multiple kinds of distributed systems, such as
Clouds, Grids or clusters. SMPI is a framework that connects SimGrid
with the MPI inter-process communication protocol: it's part of the
SimGrid project with the goal of running MPI applications in a
simulated environment. SMPI is currently under heavy development, thus
in need of constant validation. This thesis is about the
implementation of a framework that could help in orchestrating and
automatizing tests with the goal of aiding the development of the
project. SMPI validation is done by
running real-life MPI benchmarks (using possibly multiple different
implementations), as well as running the same benchmarks with
SMPI. The results can be used to check how accurate the simulation
was. Currently, this testing process consists of multiple
manually performed steps. It's tedious, repetitive work, but as
mentioned before, it's necessary. The main goal of the framework is to
provide a means to orchestrate such tests, introducing as much
automation into the process as possible, providing the possibility to
create test workflows that could be run with minimal user
interaction. With that, we could provide the possibility to speed up
the testing process, making it possible to acquire proportionally more
test results, thus helping the development team to make progress.
}
\clearpage  % Abstract ended, start a new page
%% ----------------------------------------------------------------

\setstretch{1.3}  % Reset the line-spacing to 1.3 for body text (if it has changed)

% The Acknowledgements page, for thanking everyone
\acknowledgements{
\addtocontents{toc}{\vspace{1em}}  % Add a gap in the Contents, for aesthetics

The author would like to thank his supervisor, Mark Stillwell for
the opportunity to write this thesis, as well as for his counsel and
guidance throughout the project. The author would also like to
extend his thanks to George Markomanolis and Tomasz Buchert for
providing invaluable advice and selfless support when it was needed
the most.\\[0.5cm]
Last but not least, the author wants to give a very special thanks
to his family and his fiancée for their never-ending love and
support. Without them, this thesis wouldn't have been possible to
make.

}
\clearpage  % End of the Acknowledgements
%% ----------------------------------------------------------------

\pagestyle{fancy}  %The page style headers have been "empty" all this time, now use the "fancy" headers as defined before to bring them back


%% ----------------------------------------------------------------
\lhead{\emph{Contents}}  % Set the left side page header to "Contents"
\tableofcontents  % Write out the Table of Contents

%% ----------------------------------------------------------------
\lhead{\emph{List of Figures}}  % Set the left side page header to "List if Figures"
\listoffigures  % Write out the List of Figures

%% ----------------------------------------------------------------
\lhead{\emph{List of Tables}}  % Set the left side page header to "List of Tables"
\listoftables  % Write out the List of Tables

%% ----------------------------------------------------------------
\setstretch{1.3}  % Return the line spacing back to 1.3

%% ----------------------------------------------------------------
\mainmatter   % Begin normal, numeric (1,2,3...) page numbering
\pagestyle{fancy}  % Return the page headers back to the "fancy" style

% Include the chapters of the thesis, as separate files
% Just uncomment the lines as you write the chapters

% Chapter 1

\chapter{Introduction}
\label{Chapter1}
\lhead{Chapter 1. \emph{Introduction}}

Distributed computing is a very important concept in computer
science. There are multiple types of large-scale distributed
environments that can be used for either production purposes or for
research. Parallelism is also used in a lower level, for example in
graphics processing, where we can have multiple graphics chips in one
computer to do the task.\\[0.3cm]
When doing distributed computing, communication between the processes
becomes a very important concern, since it can pose a relatively large
overhead compared to sequential problem-solving. To make parallelism
worthwhile, we have to make sure that the speedup provided by the
distribution of tasks makes up for the communication overhead. There
are two important factors to take into account when trying to
achieve this: task distribution needs to be carefully planned and the
communication protocols are needed to be optimized. Focusing on the
latter, a widely utilized communication protocol that has been under
development for many years
is provided by the MPI\cite{mpif12} inter-process, language-independent
communication API. MPI itself is just a specification, which has
multiple existing implementations. The most widely used ones include
OpenMPI\cite{ompi04} and MPICH\cite{glds96}\cite{gl96}.\\[0.3cm]
There are various reasons to why simulation, and particularly that of
large-scale computing systems has become a very
important field of research and a favored direction to go in for
scientists. Setting up a distributed environment is a complicated
endeavour: it needs both human and monetary resources. Such an
investment and commitment to one system might not be feasible - we may
only need it to run a few experiments. There is the option of renting
resources, but that also comes with extra expenses. Also, when doing
research in distributed computing, our needs possibly exceed the use
of just one single platform: we would like to test how our experiments
fare on multiple different environments. Another reason is that
sometimes we would like to predict what a given system would be
capable of, so we are fully aware of its capabilities before acquiring
it, making sure it has the attributes we need.\\[0.3cm]
SimGrid\cite{clq08} is a project providing a wide range of features
regarding simulation: it is a scientific instrument that can be
used to simulate large-scale distributed systems in order to study
their behavior by evaluating and analyzing the results of parallel
experiments run with the simulator. As mentioned before, inter-process
communication is a very important concern for these experiments. SMPI
is a framework that is part of the SimGrid project. This framework
makes it possible to simulate the execution of parallel applications
that use the MPI standard, with the simulation happening on a single
node.\\[0.3cm]
SMPI is a framework that has been validated in the past by experiment
results. Such results have been documented and published, for example
in \cite{csgscq11}. Results are needed to be reproducible, by
providing that the conducted experiments are repeatable. However,
currently, the testing process to procure such results
consists of multiple steps, many of them being manual configuration
steps, such as the allocating of nodes, the creation of a file
containing the allocated nodes or the distribution of the benchmark's
runnable between the nodes. These tedious test processes lack a
universal, user-friendly guide which could help other researchers in
reproducing the
acquired results in order to do further validation. Such a guide would
also help in repurposing, extending our experiments by changing
parameters, switching the underlying configuration, etc. Providing the
possibility to easily repeat the experiments conducted for a paper
is very important: our goal when writing a paper is not only to
announce our results but also to convince our readers that our results
are correct.\cite{m10} The best way to prove that we are right is if
we provide a straightforward way for anyone to repeat our experiments,
so they can see for themselves that our results stand. Also, if other
scientists are unable to repeat our experiments, that means that they
are unable to incorporate our results in their research (which may or
may not be related to our field). This would the main point of
scientific collaboration and as such, should be facilitated as much as
possible.\\[0.3cm]
SMPI is a well-documented and working framework, but also an active
project. It is also a project with the goal of applying simulation to
study large scale computing systems, which is a relatively new
concept. As such, the project is under constant
development. Currently, a lot of resources are spent on developing
generic network models for
the simulator. Extensive testing is needed for the development
process, since continuous validation is necessary to see whether or
not we are heading in the right direction, whether or not the
simulator correctly represents the real-world behavior. This
testing process, as previously mentioned, is currently very
time-consuming. This doesn't only limit the number of tests that can
be run, but also limits the reproducibility of the
results that are achieved with SMPI, the importance of which has been
discussed in the previous paragraph. By constructing a framework that
simplifies the testing process, more reliable and verifiable results
could be produced, as well as it would make the SMPI project members'
lives easier. It is important to include as much automation in the
framework as possible, since other researchers that want to repeat the
experiments might not be computer experts. Our processes can't be
fully automated though. For example, certain parameters, such as the
runnable, or the number of nodes to allocate have to be set - but,
when doing the implementation, we aim to keep such non-automated steps
minimal.\\[0.3cm]
This thesis discusses how such a framework could be built and provides
an implementation, utilizing the XPflow\cite{bn12_2} experimentation
engine. XPflow is a fairly new project, constantly under development.
Although it hasn't been officially released yet, it's fairly stable
and suitable for our project. It's designed to help automating
experiments, utilizing a top-down approach taken from business
workflows.\\[0.3cm]
With the implementation, our main goals are to make testing faster and
more efficient by providing a way to automate the process. Linking
together multiple different MPI experiments could further help
speeding up the course of getting results. Another
important goal is to take the burden of doing the currently
repetitive, error-prone process over and over again off the developers
in the SimGrid team. The framework can also prove to be a means to
have a better organization of tests - separate test files could be
written and stored in their respective directories, categorized
according to what they can be used for, instead of the current
on-the-fly manual process.\\[0.3cm]
The thesis is organized in the following way. In Chapter 2, we go over
research material connected to the thesis subject, citing relevant
documents. In Chapter 3, we examine various aspects of the problem
that the implementation of the framework poses. While Chapter 3 is
more of a general problem description, Chapter 4 is about specifics,
implementation details of the framework. Then in Chapter 5, we
discuss the process of evaluating the framework, which will be done by
running the same experiment in the "old", manual way and with the
framework. In the same chapter, we observe how the actual evaluation
goes, observing the steps in both cases, then citing the numerical
results. In the end of the chapter, we compare these results, as well
as the two methodologies to get to a conclusion whether or not we
reached our goals. Then finally, in Chapter 6, we conclude the thesis
with a summary of the content, and in the end, we take a look at what
the possible directions for development are for the project.
 % Introduction

% Chapter 2

\chapter{Literature Review}
\label{Chapter2}
\lhead{Chapter 2. \emph{Literature Review}}

In the literature review, we discuss the area of research covered by
the thesis, citing relevant papers and articles that serve as a base
of ideas for this document.\\
First, we talk about the Message-Passing Interface (MPI), the parallel
programming API used for the purposes of this thesis. We also talk
about the different implementations of this API, notably OpenMPI and
MPICH. We discuss the methods of modelling and simulation in general.
Then, we talk about SimGrid, which is a simulation-based
framework, and SMPI, the implementation of MPI that runs on top of
SimGrid. While discussing SMPI, we present the idea of a framework
that would make it possible to automate running tests, trace
collection and post-processing. In connection to this framework, we
talk about workflows as a means to automate our experiments and
whether scientific or business workflows are more suitable for
us. We also talk about the importance of reproducible research, which
is another argument for constructing such a framework.
\section{MPI}
Distributed computing is a very active and important subject of research
in computer science, including fields such as cluster computing, grid
computing, Cloud computing, or peer-to-peer computing. Communication
between the different processes in a distributed application can be
implemented in a number of ways. As communication is necessary in most
cases, a standardized communication protocol can be a lot of help when
developing a distributed program. The Message-Passing Interface (MPI)
is a language-independent message-passing library
interface specification. It is not a language, but a standard - there
exist multiple MPI implementations. Since its take-off, it has become
a de facto standard for inter-process communication. The standard
provides vendors a clear set of routines, that they can implement
efficiently, or in a way that it suits the hardware they
provide.\cite{mpif12}
\subsection{OpenMPI}
OpenMPI is an MPI implementation with the goal of being able to
achieve good performance on a wide range of different aspects of
high-performance computing. To efficiently support multiple types of
parallel machines, high performance “drivers” for all established
interconnects are developed. These include TCP/IP, shared memory,
Myrinet, Quadrics, and Infiniband. Features for checking data
integrity are provided in order to account for network transmission
errors. With the utilization of message fragmentation and striping
over multiple (potentially heterogeneous) network devices, OpenMPI
provides an increased bandwidth to applications, as well as the
ability to handle the failure of network devices during
runtime.\cite{ompi04} On the Grid'5000 cluster, which is used to most
of the research conducted for this thesis, OpenMPI is the default MPI
implementation used by the default images.
\subsection{MPICH}
MPICH was originally developed during the MPI standards process
starting in 1992 to provide feedback to the MPI Forum on
implementation and usability issues. This original implementation was
based on the Chameleon portability system to provide a light-weight
implementation layer (hence the name MPICH from MPI over
CHameleon). Around August 2001, development begun on a new
implementation called MPICH2.\cite{mpich12} This implementation
introduced improvements on collective communication operations by
using multiple algorithms, choosing between them depending on certain
variables - for example the message size.\cite{trg05} Another
important result during the development of MPICH2 was the design of
the Nemesis communication subsystem and the porting of MPICH2 on that
system. The efficient implementation of shared-memory communication
helped Nemesis MPICH2 achieve low latency and high
bandwidth.\cite{bmg07}\\
Starting with November 2012, the project is renamed to MPICH, with
version number 3.0.\cite{mpich12}
\section{Modelling and Simulation}
In distributed computing, modelling means creating an abstraction of a
real system by taking only the aspects of it that are relevant to the
system's behavior into account. Once constructed, such a model becomes
a tool with which we can investigate the behavior of the
system.\cite{h12_1}
\subsection{Advantages of Modelling}
Modelling and simulation techniques have been used extensively in
parallel computing and is an ongoing research topic, with new
challenges continuously arising. There are various reasons for its
importance.\\
Conducting experiments on real-world systems can be
infeasible because experimenting would disrupt the service that is
provided by the system. For example, in the case of a mail server,
experiments or monitoring could cause delay, or maybe even data
loss. Service disruption can sometimes be even dangerous, in addition
to being an inconvenience: in the case of a nuclear reactor, delay or
loss of data can prove fatal. Timeliness can be as important in such
systems as correctness. However, performance analysis and monitoring
might be crucial to draw conclusions about maintenance, for
example. Another problem with direct experimentation is that the
information we are looking for may not be available, or may be
complicated to get. For example, in most operating systems, it is
difficult to obtain the exact timing of instruction-level
events.\cite{h12_1} Also, when conducting experiments on a real-world
system, results are often non-reproducible, due to resource
dynamics.\cite{clq08} Another argument on the side of modelling is
that it provides the ability of experimenting on different
configurations. Investing in a large-scale computer cluster, or the
setup of a distributed grid environment is an expensive and tedious
process. Investors want to make sure that they get what they
want: they impose performance constraints on the system. This means
that they want to know how the system will behave
before buying it and setting it up. To predict the behavior,
experiments are needed to be conducted. We need to do these
experiments on different setups, before finding out which one is the
best in the current situation. Changing the hardware or software
configuration parameters on a real-world system is very inconvenient -
in most cases, it's not doable, because of time and money
constraints. Thus, the solution is to simulate the desired
system, and run the experiments there. This way, changing the
configuration is simple and costless.\cite{h12_1} Another great
benefit of simulation is that in a classroom setting, students can
learn the principles of high-performance and distributed computing
without actual access to a parallel platform.\cite{csgscq11}
\subsection{Analytical and Simulation Models}
The accuracy of a model can vary: we can make an analytical, or
qualitative model, in which all definite values are abstracted away -
in this case, we get a representation of
the system, which can be analysed mathematically to deduce its
behavior. When using this method, no experiments can be conducted, we
solely rely on theoretical analysis. In contrast, a simulation model
is a stochastic model, which is an algorithmic abstraction of the
real-world system that can be executed to reproduce the system's
behavior. This model is also called a quantitative model, as we can
get estimates of the modeled system's quantitative attributes, such as
response time or throughput. In other words, we can use a simulation
model to conduct performance analysis on a system, without actually
having the actual system at our disposal.\cite{h12_1}\cite{h12_13}\\
When wanting to get a prediction about how a specific system would
perform, a theoretical model, in most cases, produces unreliable and
unrealistic results - it's not feasible for such accurate
predictions. The vast majority of research results are obtained via
empirical evaluation of experiments.\cite{clq08} For these reasons, we
use the simulation model in this thesis. As we stated before, such a
model can be executed, which is called simulation. During simulation,
the model is supposed to behave like the real system would. It is hard
to produce a 100\% accurate simulation, but more and more reliable
solutions are being developed. The simulation model contains
more aspects of the real system compared to the theoretical model, in
order to accurately represent the system, while still avoiding
unnecessary detail.\cite{h12_1} Creating and executing a simulation
model is complicated, computationally expensive and poses a number of
challenges, thus, a good simulation framework (such as SMPI) can prove
to be of much help when conducting experiments.
\section{Off-line and partial on-line simulation}
Full simulation - including CPU and network emulation - of a parallel
application can be, in many cases, even more resource-intensive than
running real-world experiments. This contradicts the fact that one of
the most prominent goals of simulation is to observe the behavior of
such large-scale platforms that aren't available. Thus, there is much
interest in more efficient simulation approaches. \cite{bdglmqssv13}
The most widely used of such approaches fall into two categories:
off-line simulation, which is also called trace-based or post-mortem
simulation and on-line simulation, which is simulation via direct
execution.\cite{csgscq11} As in the subject of this thesis, we are
interested only in the simulation of MPI applications, we describe
the two different simulation approaches concentrating specifically on
that subject.
\subsection{Off-line simulation}
For conducting off-line simulation, logs or traces are needed to be
collected of an execution of the MPI application to be simulated,
taking place on a
real-world platform. This is necessary because the obtained traces are
used as an input for the simulator, which then replays the execution
traces as if the application was running on the target platform. This
platform's characteristics may differ from the one's that we obtained
the traces from, since we may want to use the simulator to predict the
application's performance on a different system. Thus, there is a need
to calculate how the target platform would execute the application,
based on the traces we got on the other platform. The typical approach
to this problem is to first compute the time intervals between the MPI
communication operations. During these intervals, local computations
were conducted, that's why we call these "CPU bursts". During
simulation, we have to account for the differences between the
performance of the platforms by modifying the time these CPU bursts
take. This can be done by simply scaling the time intervals, or by
using more sophisticated methods, by calculating exactly how the
application's computational signature and the platform's hardware
signature relate.\cite{csgscq11} Communication operations, of course,
also need to be simulated. This is done based on the events recorded
on the trace, and on the network model of the simulated
platform.\cite{csgscq11}\\
As mentioned in \cite{csgscq11}, there are multiple downsides and
challenges to the off-line approach. On such downside is that when
wanting to simulate a relatively larger-scale application, the size of
the obtained traces can be so large, that running the simulation on a
single node might become a problem. Methods in order to overcome this
obstacle include a compact representation of the traces in order to
reduce its size. Another solution is to only consider a carefully
selected subset of the obtained traces. A big disadvantage when using
off-line simulation is that because we use the traces as an input to
the simulator in order to replay the execution of the application, the
simulation is dependant on the platform we collect the traces
on. This means that, for example, there can be features in the
obtained traces that might not be available on the target platform. In
most cases, it is also necessary that the two platforms have the same
number of nodes to run the experiment on. Although there has been a
good amount of research done in the area, MPI itself and also the
application might alter its behavior depending on problem and message
size. Because of this, simulating the scaling of an application is a
very hard, if not impossible task.\cite{bdglmqssv13}
\subsubsection{Time-independent traces}
Another link that ties the produced trace to the host platform occurs
when we use timed traces, meaning that each traced event is associated
to a time-stamp. Since the time delays between the events are specific
to the platform specification, the simulator has to apply a correction
factor to these delays when running the simulation on the target
platform. Thus, the simulator has to know precisely the specification
of both the host and the target platform, in order to be able to
calculate this correction factor. Another difficulty regarding that
comes up regarding this problem is that actually calculating the
correction factor is a tedious process. It can take a considerable
amount of time, depending on how similar the host and the target
platforms are.\cite{dmsq11}\\
In \cite{dmsq11}, a solution to this problem is proposed:
time-independent traces. Acquiring time-independent traces means that
the traces won't contain any timestamps, breaking this link between
the acquisition and the replay of the traces. In these type of traces,
for each computation or communication operation, we log the volume of
the operation (in number of floating point operations or bytes)
instead of the time the execution took. This type of information, in
most cases, does not vary depending on the platform the experiment is
run on. The exceptions are the adaptive MPI applications that modify
their execution path according to the execution platform.\\
\cite{ms11} contains a guide describing how to acquire such traces on
the Grid5000 platform. The guide was used to serve as a base for the
process on how to acquire traces. Since the work in this thesis is
mostly related to producing traces for validating on-line simulation,
in which case time-stamps don't have an influence on the process
(neither in a positive, nor a negative way), the extraction of
time-independent information from the traces can be omitted in our
case.
\subsection{Partial on-line simulation}
Partial on-line simulation is a different approach. Here, we execute
the program with no or very little modification on a host platform,
that tries to mimic the behavior of the target
platform.\cite{csgscq11} Computational tasks are executed on the
hardware, but the timing and the delivery of the messages is
calculated by the simulation environment. Thus, the simulator is
responsible for mainaining the correct order of the events, both
computational and communicational.\cite{bdglmqssv13}\\
A downside of the on-line approach is that since we actually execute
the code, the resource needs for running the simulation is about as
high or even higher (in case of needing an extra node to run the
simulation component, for example) than it is for the actual
experiment. Techniques have been implemented in order to help
alleviate this problem. The basic idea is that
the actual results of the experiments (for example, the result of
multiplying two matrices) might not be important in our case: we are
only interested in the \emph{time} it takes to get those results on
the target platform. This is why methods can be employed which trade
off accuracy for performance. This idea might not be feasible for
experiments where data-dependent application behavior is
vital, but a large portion of benchmarks can be indeed simulated
this way, providing a reasonably accurate execution profile.\\
Although slower, on-line simulation is more general than the off-line
approach, as it does not, in any way depend on some other platform -
whereas in the case of off-line simulation, as we mentioned before,
the trace is acquired on a different platform, with maybe specific
application configurations, thus inevitably bringing dependencies.
\section{SimGrid}
For reasons mentioned before, simulation techniques have historically
been widely utilised in several areas of computer science,
e.g. microprocessor design, network protocol design. Due to this, a
lot of effort went into developing the technology and as a result,
widely used and reliable simulation frameworks have been developed in
these areas. However, there hasn't been a well-developed standard
simulation tool for what we talk about in this thesis: execution of
distributed applications on distributed computing platforms. Rather,
there has only been a number of in-house developed, highly specialized
tools to satisfy the need of the community. SimGrid is a more generic
simulation framework that is being developed to be one of the
acknowledged and widespread tools for simulation in large-scale
distributed computing.\cite{clq08}\\
SimGrid's key features include:\cite{clq08}
\begin{itemize}
\item A scalable and extensible simulation engine that implements
  several validated simulation models, and that makes it possible to
  simulate arbitrary network topologies, dynamic computational and
  network resource availabilities, as well as resource failures;
\item High-level user interfaces for researchers (who are not
  necessarily computer science experts, but rather experts on their own
  field of research) to quickly assemble simulation prototypes in either
  C or Java;
\item APIs for distributed computing developers to create distributed
  applications that can run seamlessly in either "simulation mode" or
  "real-world mode", in order to be able to test it on the simulated
  environment before actually deploying it.
\end{itemize}
SimGrid is a very active project, both in terms of research and in
terms of development. It is a favored tool by researchers, which is
proven by the increasing number of papers written where the research
was conducted using SimGrid as a scientific instrument. In terms of
development, the developer team envisions a number of directions for
future work: addition of a model for disk resources; extention of
scalability to improve usability in the P2P domain; ability to
dispatch simulated nodes over several physical machines.\cite{clq08}
Another important field of research for the SimGrid team is the
implementation of the API that has already been mentioned: the
Message-Passing Interface (MPI).
\section{SMPI}
As stated before, MPI is one of the most widely used APIs for
communication between nodes in distributed computing. SMPI is a
framework for simulating on a single node the execution of parallel
applications implemented using the MPI standard. It is part of the
SimGrid project and as such, it is built on the SimGrid simulation
kernel, benefiting from its fast, scalable and validated network
models. SMPI also extends the existing model with other techniques,
such as a validated piece-wise linear model for data transfer times
between cluster nodes. SMPI simulations also account for network
contention - timing and delivery of the messages are determined using
the network model of SimGrid.\cite{csgscq11} A current limitation in
SMPI is that it is unable to simulate high-performance networking
hardware such as Infiniband. Thus, when wanting to compare simulation
to real-life results, we have to make sure those results were gatheres
using Gigabit ethernet.\\
Three of the main challenges for simulating an MPI application are:
\begin{itemize}
\item Accuracy: The prediction of the real-world execution time (the
  "simulated time") needs to be as accurate as possible, so that
  reasonable conclusions can be drawn from the experiments.
\item Scalability: We want to be able to simulate large-scale
  applications within a reasonable timescale.
\item Speed: It would be advantageous if the simulation time (the
  actual time of running the simulation) would be as low as possible,
  compared to the simulated time (the predicted execution time of the
  real-world application).
\end{itemize}
As for simulation methods, SMPI can be used for both off-line and
on-line simulation, although the emphasis is more on the on-line
approach, since it's actually a partial implementation of the MPI
standard in itself, thus making it feasible for executing MPI
experiments. More specifically, in SMPI, the goal is to be able to
make such simulations on a single node. The most prominent challenges
when doing this are the large CPU and memory requirements. SMPI
provides some special techniques that help overcoming these
challenges. The basic idea about trading off accuracy for performance
has already been described in the previous section about on-line
simulation. SMPI implements multiple such techniques, allowing to run
experiments with such high resource requirements that would otherwise
be impossible to fulfill. Such a method in order to reduce CPU usage
is to run the
benchmark only on a subset of all the nodes, while in place of running
the code on the others as well, we just insert the computation time
that we got previously. Apart from CPU usage, we need to also account
for the need for memory. A technique for that is "RAM folding": here,
multiple simulated processes, that in SMPI are, in fact, simple
threads, use the same reserved memory location, thus overwriting each
other's data structures. Also, another implemented solution is to
remove large data array references from the code, with the help of the
compiler which can result in the complete removal of potentially
large, now unreferenced arrays. Again, this obviously corrupts the
results that the experiment program gives, but in the same time helps
to simulate applications that would use such an amount of memory that
just wouldn't be physically possible to provide in our testing
environment, while still providing a reliable estimate of the
performance.\cite{bdglmqssv13} These features are disabled by default,
they have to be explicitly enabled by the user.\\
Extensive testing was conducted in \cite{csgscq11} to verify the
previously mentioned qualities of the framework. In these tests, the
OpenMPI and MPICH implementations were used to serve as verification
benchmarks: the same experiments were run using both MPI
implementations, as well as simulated with SMPI. The
results show that SMPI predicted the execution time of OpenMPI and
MPICH applications for point-to-point, one-to-many and many-to-many
applications with an average error value of under 10\% in each
cases. Using the aforementioned techniques to reduce the memory
footprint, SMPI tests
were successfully conducted on a scale of up to 448 processors. The
results showed that the predicted execution times were
underestimates with an average error value of 18.5\%, which is higher
than in previous experiments without these techniques. We have to note
here, though, that
certain tests weren't successful without the RAM-folding techniques,
due to an out-of-memory error. This shows, that although it poses
difficulties, reducing the memory usage is vital in SMPI.\\
As SMPI is an actively developed project alongside SimGrid, there are
a number of research directions. One major development to the
project would be a testing framework that would aim to lessen the
burdens of testing as much as possible. The goal is to provide a
unified method to set up experiments across different environments and
to do it with as little necessary adjustments on user part as
possible.
\section{Experimentation tools}
Experiment orchestration and process automation is not a new
idea. There exist multiple tools for doing this. Among others, such
tools are Expo\cite{vr08}, Plush, OMF\cite{rosj09} or the
Grid5000-specific g5k-campaign. The problem is that most of these
tools use a bottom-up design, meaning that in order to understand and
use the experiments, the user has to understand the lower-level
building blocks first. A top-down approach would make it possible to
start the design of the experiment with a high-level description, then
work our way down to the lower level details while implementing. There
already exists an approach like this, in Business Process
Management (BPM).\cite{bn12_2} Before talking about BPM though, let's
take a glance at workflows in general.
\section{Workflows}
When talking about a framework to automate MPI experiments, we are
essentially talking about creating workflows: we connect multiple
steps, making it possible to execute them in a chain, with no or
minimal user interaction during the process. In the application level,
workflows are abstract in the sense that the workflow only describes
the goal of the experiment, its components and its dependencies. Lower
level implementation details are hidden from the user of the
workflow. This provides the possibility of changing the implementation
without having to change the high-level workflow description - as long
as the new implementation still has the properties that are written in
the description.\cite{dssbgkmvbgljk05}
In the subject of workflows, there are two main approaches that we
will discuss: scientific workflows and the previously mentioned
business workflows.
\subsection{Scientific workflows}
Scientific analyses often have to be conducted in several different
hardware and software environments. Exporting and importing data from
and to different environments can be a tedious task, slowing down the
work process, forcing researchers to divide their efforts between
computation management and their actual research. This is the main
reason scientific workflows are widely used in various different
scientific domains: they are a formalization of the ad-hoc
process that a scientist has to go through in order to get from raw
data to publishable results. Since the raw data to be analyzed can be
large, heterogeneous and complex, the process can be computationally
intensive and produce derived data formats, which is one of the main
differences between scientific and business
workflows.\cite{abjjlm04}\\
There are several tools that help in experiment design, mapping of
computing resources to the workflow and handling exceptional
situations. Some of the more well-known tools include
Kepler\cite{abjjlm04}, Pegasus\cite{dssbgkmvbgljk05} and
Taverna\cite{whfwwsdnfbbbhnvsg13}.\\
Scientific workflows are well suited for managing computation on
\emph{a priori} available data or data queried from public
databases. However, it's not well suited to cases when data
acquisition is actually part of the workflow process, or in other
words, when the source of the data is the computer system
itself. They are also data flow-driven, which is not true in the case
of our processes that we want to automate. This makes scientific
workflows not optimal for research conducted in computer science.
\subsection{Business workflows}
Business workflow management systems are usually based on agreed-upon
standards in order to facilitate communication between different
software systems and companies. The workflow logic is control
flow-driven and includes constructs to specify paths and
conditions.\cite{skd10} The top-down approach it uses is what can make
it viable in our case: in Business Process Management, the first step
is to understand the organization. Then we can model its processes as
workflows and execute and monitor them. While monitoring,
we can spot defects and work out ways to improve the organizational
activities, as well as we can redesign the processes to make them
cheaper and faster.\cite{bn12_2}\\
This approach can be utilized in the domain of experiment
orchestration, making business workflows a viable choice when
approaching our problem. An experimentation engine with the goal of
implementing this idea is XPflow\cite{bn12_2}. This engine was used to
implement the test automation framework in this thesis and will be
discussed in a bit more detail later on in this document.
\section{Reproducible research}
New scientific ideas, developments and results are only useful when
they are documented and published. It is vital that results are
announced, so others can be aware of the latest developments on their
field of research. This helps in creating a linked data cloud, used by
scientists to incorporate various output of other research into their
own, using previous results as "stepping stones" to achieve something
new.\cite{babbccrddg10} But simply publishing results is not enough in
order for others to make use of them. Besides announcing the
achievements, the other goal of scientific publications is to convince
the readers that the results it presents are correct. Besides
theoretical reasoning, papers in experimental science should provide a
documented methodology describing how the author has gotten to those
results.\cite{m10} The methodology has to be detailed and precise
enough so other researchers can repeat the same steps, thus
reproducing the same results. This is vital in order to provide the
possibility to verify those results and to fully understand
them. We have to note that in a reproduced experiment, it's not the
\emph{raw results} that need to be identical: they merely need to fit
within a statistical margin, compared to the original results, so that
the \emph{conclusions} derived from them can be the same.\\
Reproducing the results also makes for a starting point for
further development, as the described methods used for reproduction
can be extended to achieve something more or something different in
the same area of research, or repurposed to gain useful results in a
completely different area. Reproducibility is a relevant concern in
the case of SMPI and one of the main goals of the testing and
validation framework is to make developments in the area.\\[0.5cm]
In this chapter, we reviewed the background material to the general
concepts related to this thesis. We talked about the MPI inter-process
communication API and its two different implementations: OpenMPI and
MPICH. Then we talked about modeling and simulation in general and how
simulation is advantageous in certain situations. Relating to
simulation, we talked about SimGrid, a multi-purpose simulation
framework. Then we talked about the project that interconnects these
concepts: SMPI, which is a tool that is able to simulate MPI programs
on SimGrid. We talked about its features and how it's been previously
validated with extensive testing. Since testing is currently a tedious
multi-step process and continuous validation is necessary, the project
is in need of a test automation framework to help with the development
process. Also, such a framework would make documented experiments
repeatable, as well as the results reproducible, which is an important
concern as well. The subject of this framework is connected to the
subject of workflows, which we discussed in some detail in the section
after. We discussed workflows in general, as well as scientific and
business workflows. We concluded that business workflows are more
suitable for our current problem because of its top-down, control
flow-driven approach. We briefly mentioned the XPflow experimentation
engine, which is based on this approach and is utilized in
implementing the framework. In the end, we also discussed the
importance of reproducible research, which is another important aspect
in our motivation for wanting to create the automation framework.
 % Literature Review

% Chapter 3

\chapter{Problem Description}
\label{Chapter3}
\lhead{Chapter 3. \emph{Problem Description}}
 % Problem Description

% Chapter 4

\chapter{Implementation}
\label{Chapter4}
\lhead{Chapter 4. \emph{Implementation}}

\section{An example process}
To reiterate the more detailed description of trace acquisition, an
example process that we would like to automate consists of the
following steps, assuming that we already have a customized image set
up for our tests, as well as we already have an executable benchmark
application we'd like to run:
\begin{itemize}
\item allocate the specified number of nodes on a specified site
\item deploy our custom image on the allocated nodes
\item create a nodefile containing the names of the allocated nodes
\item broadcast the runnable across the nodes
\item disable all cores but one on every allocated node (see 3.4.1 for
  explanation)
\item run the mpi experiment from a chosen "head" node (can be any of
  the allocated nodes)
\item gather the traces from the allocated nodes to the head node
\item merge the traces
\item convert merged trace file to Paje format
\end{itemize}
As a side note: in this example, we assume that we are working on the
Grid'5000 testbed. As previously mentioned, most of the work regarding
this thesis was conducted there.\\
As for the example process: parameters that the user can give to it
are the number of nodes, the chosen Grid'5000 site, the image to
deploy, the runnable and parameters to mpirun (for example to disable
Infiniband) and the trace\_gather script. The nodefile created in the
3rd step is necessary for the operations that are needed to do
something on all the allocated nodes.\\
Now let's take a look at the experimentation engine we'd like to use
for implementing the framework to automate processes like this.
\section{XPflow}
In 2.8.2, we already discussed business workflows, foreshadowing the
fact that XPflow, the tool used to implement our framework, is based
on that approach, which includes Business Process Modeling and
Management. There are 2 main concepts in XPflow\cite{bn12_2}:
\begin{itemize}
\item Process: It is the high-level description of the experiment,
  written in a DSL, which is embedded in Ruby. Processes are
  responsible for orchestrating activities and other processes,
  creating a workflow.
\item Activities: The low-level building blocks of the
  experiments. They are written in Ruby and used for implementing the
  lower-level details of the experiment, to do the "real work" in it
  (for example: manage files, start the MPI job).
\end{itemize}
 % Implementation

\externaldocument{Chapter4}
\externaldocument{Chapter3}

% Chapter 5

\chapter{Evaluation Plan}
\label{Chapter5}
\lhead{Chapter 5. \emph{Evaluation Plan}}

\label{sec:evaluation_plan}
In this chapter, we go over the process of the evaluation of the
implemented framework, the goal of which is to make
conclusions about whether or not we succeeded in achieving the
requirements we specified earlier. The evaluation will be done by
running a multi-step, fairly typical experiment both manually and with
the framework, so we can easily observe the differences.\\
The experiment process used in the evaluation will consist of the same
steps as mentioned before in the Implementation chapter as an example
MPI experiment process (see \ref{fig:xpflow_example2}), used to collect
RL traces, which can be used for example for the development of
SMPI. The steps in the process are very generic though: we need to be
more precise about certain details of the experiment.
\section{Experiment specification}
\subsection{The benchmark}
The benchmark used for evaluation is taken from the NAS Parallel
Benchmarks suite\cite{jfy99}, which is a small set of benchmarks
designed for performance testing of parallel systems. The benchmark
chosen for our current purpose is called \emph{lu.B.8}. The name
consists of 3 parts. The first part, \emph{lu} indicates what the
experiment is about: as its name suggests, the LU benchmark solves a
system of equations represented with a matrix, with the LU
factorization method.\\
The second part of the name, \emph{B} is an
indication about the complexity of the problem that the benchmark
solves. The NPB suite defines so-called "problem classes" for its
benchmarks. B is in the middle "standard" complexity category, being
more complex than the Small (S), the Workstation-size (W) and A, the
least complex "standard" problem size, but less complex than C and the
larger test problem sizes.\cite{d13}\\
Finally, the number \emph{8} at the end of the benchmark's name
indicates how many parallel processes it is intended for to be solved:
for our purposes, we choose it to be 8, thus, we will allocate 8 nodes
for our experiment, each of them running 1 MPI process to solve the
problem.\\
It's important to note that when doing the experiment, we assume that
the benchmark is already compiled, in a specified place in the user's
home folder: /home/dlehoczky/NPB3.3/NPB3.3-MPI/bin/lu.B.8. For the
user \emph{dlehoczky}, the environment variable \emph{\$NPB\_DIR} is
set to /home/dlehoczky/NPB3.3/NPB3.3-MPI, thus it can be used as a
shortcut when accessing the benchmark. Also, it is assumed that
ssh-keys are set up correctly so the user can reach the specified
Grid'5000 site without any password.
\subsection{The environment}
For our experiment, we use the Grid'5000 testbed, discussed in greater
detail before, in \ref{sec:environment}. As we said before, there are
many different sites to choose from. During development,
many tests were needed to be done. Sometimes one or two sites were
undergoing maintenance or became unstable for various reasons. Towards
the end of the development process, the Lille site proved to be
reliable in terms of availability, this is why it was chosen as the
site to run our final tests on.
\subsection{The OS image}
\label{sec:image}
We use a customized image called
\emph{wheezy-x64-big-lehoo}. This image contains all the necessary
tools required to run the described process. These tools were
mentioned before, at \ref{sec:rl_traces}, when we talked about how to
collect RL traces, which this experiment is about. To reiterate, let's
sum up what our customized image contains in order to execute our
experiment process:
\begin{itemize}
\item OpenMPI 1.6.4;
\item the \emph{TAU}\cite{sm06} profiling tool;
\item the PAPI\cite{mbdh99}\cite{lmmsl01} interface to low-level
hardware counters (configured so it's linked to TAU, which uses it for
tracing);
\item the \emph{Program Database Toolkit (PDT)}\cite{lcmsmrr00} (also
configured to be linked to TAU);
\item the \emph{trace\_gather}\cite{ms11} MPI program;
\item \emph{Akypuera}\cite{s13}, a library to trace mpi applications
and generate paje trace files. (In our experiment, we only use
its \emph{tau2paje} trace converting script.)
\end{itemize}
\subsection{The experiment process}
\label{sec:experiment_process}
As mentioned before, the experiment process will consist of the same
steps as in the example discussed before
(see \ref{fig:xpflow_example2}).\\
To prepare our experiment process, we start by
logging in to the frontend with our user. Then, we start an
interactive job, allocating the desired number (in our case, 8) of
nodes for a specified amount of time. We won't specify what nodes we
want, the system will decide which ones we get. Since there might be
performance disparities between two different sets of nodes, we make
sure to use the exact same set of nodes in the two experiments (the
one done manually and the one done with the framework). We do this by
running the experiment with the framework first, then connecting to
the job started by the framework and repeating the experiment
manually, starting from the deployment part. The node allocation is
not really an important part of the experiment, thus, it's not a
problem that it's omitted from the manual execution.\\
As part of the process that we conduct in both experiments, first, we
deploy the previously mentioned (\ref{sec:image}) customized image on the
nodes. Then we broadcast the runnable (\emph{lu.A.8}) to every node's
/tmp directory. After that comes the step where we disable all but one
core on every node. The reasoning for this step has been discussed
previously (see \ref{sec:multiple_cores}). This concludes the
preparation stage.\\
After that comes the execution of the benchmark. We use OpenMPI 1.6.4,
installed on our image.\\
After the running of the benchmark, comes the post-processing. If we
compiled our benchmark correctly with TAU, one trace file (.trc)
and one event file (.edf) is generated for each MPI process. The
traces are generated on the node of execution. This is why first,
we run the \emph{trace\_gather}\cite{ms11} script to collect the traces
and event files scattered across all the allocated nodes to the head
node. When we have all the files on our head node, we use
TAU's \emph{tau\_treemerge.pl}, which is a script that merges all our
trace files and event files into one trace and event file
respectively, also trying to account for any clock skew
(see \ref{sec:clock_synch}) between the trace files with
post-processing methods. Finally, when we have one merged trace and
one event file, we can convert our TAU trace to a format that is
compatible with the Pajé\cite{cob00} visualization tool. Visualizing
the traces makes it easier to analyze and compare. For the conversion,
we use Akypuera's\cite{s13} \emph{tau2paje} script.\\[0.5cm]
In this chapter, we went over the specifics of the evaluation of our
implementation. First, we talked about the chosen benchmark to gather
traces from, which is called \emph{lu.B.8}. We alse explained there
what the name stands for: LU indicates the problem the benchmark
solves, B is the problem class and 8 indicates how many nodes the
benchmark is compiled for. We also discussed what the chosen
environment is going to be (Grid'5000) and what operating system image
we are going to deploy on our nodes before running the benchmarks on
them. Finally, we went over the experiment process: the RL trace
collection for a given benchmark. We mentioned that our experiment
process will be done both with and without the framework to
demonstrate how the two methods differ from each other. We discussed
the preparation steps, the running of the benchmark and the
post-processing part. Now, in the next chapter, we'll take a look at
the results of our experiments.
 % Evaluation

\externaldocument{Chapter5}
\externaldocument{Chapter4}
\externaldocument{Chapter3}

% Chapter 6

\chapter{Conclusion}
\label{Chapter6}
\lhead{Chapter 6. \emph{Conclusion}}

\section{Summary of work}
SMPI is a framework for single-node simulation of parallel
applications using the MPI standard. It is a relatively new project
that is under heavy
development currently - a generic network model is under development by
the project team, thus needing continuous validation with test
results. The main goal of this thesis was to create a framework, which
could be used to automate MPI experiments, in order to alleviate the
burden of doing the current manual testing process from the
developers. The other main benefit of the framework would be that
proportionally more tests could be run, satisfying the constant need
for validation more easily, thus facilitating a speedup in the testing
process.\\
An implementation of such a framework has been developed through the
course of making this thesis. For implementation, the XPflow
experimentation engine was used. XPflow takes its idea of top-down
approach from business process management: the main point is that
first, we start out with a high-level description of the experiment,
working our way down to the low-level details. It is a fairly new
project, so much so that it hasn't officially been released
yet. Nevertheless, it has proven to be reliable enough to serve as a
base for the implementation.\\
The author of this thesis mainly used the Grid'5000 testbed for
development and testing purposes. Grid'5000 is a multi-site platform,
with its sites residing different places across France, each site
hosting multiple different clusters. The platform has a user-friendly
API which can be used to allocate nodes for a given time, deploy
operating system images on them, which can then be customized and
saved for later use. XPflow also provides its own plugin to use
Grid'5000. The testbed proved to be suitable for the purposes of this
thesis.\\
The evaluation of the implementation has been done by running an MPI
trace collection experiment both manually and with the framework. The
chosen benchmark was the LU benchmark from the NAS parallel
benchmarks. After running the experiments, we found that although
there were differences both in running time and in the traces itself,
essentially the results were the same: running time differences were
mostly in the operating system deployment part of the experiment,
which can be accounted to the availability of Grid'5000 resources, but
the benchmarks were running for almost exactly the same time; and as
for the traces, the differences lied in the timings and the order of
the MPI operations, the cause of which is most probably simply the
arbitrary nature of distributed experiments - but the runtime, as well
as the workloads of the appropriate processes were the same.\\
After comparing the results, we compared the two methodologies using
other aspects, in order to make a conclusion whether or not the goal
that was set at the beginning was reached. And although there is much
room for development (see next section), we can conclude this thesis
on a positive note: most of the set goals have been reached. The
tedious, repetitive, error-prone manual testing process has been
largely automatized, with minimal user interaction. We are now able to
create reusable experiment code, providing the possibility to run
certain tests regularly, or more than once in a row, possibly with
certain modifications. Checkpointing also makes for a very handy
feature, since we can reuse existing jobs to save ourselves even more
time. There is also the possibility of creating higher-level
workflows by conducting more tests in a row, thus achieving a
considerable speedup in producing test results.

\section{Development directions}
During the writing of this thesis, the framework has been developed to
a version where running MPI tests and collecting its traces, such as
it was done for the evaluation can be reliably done. However, there
are a lot of work that could possibly be done to develop it to be
more generic and more comfortable when orchestrating experiments.
\subsection{Configuration file}
Currently, the framework saves certain variables after it was given as
a parameter to a method call. This can sometimes lead to confusion as
to whether or not it was already given, or if it is set correctly. It
would be more straightforward method to provide the user with
the possibility of creating a configuration file with some of the most
important parameters, preferably ones that are relevant throughout the
whole experiment, such as deployment information (site used,
number of nodes to reserve, image used, etc.) or paths to certain
runnables. Parameters such as paths to the benchmarks are more prone
to change between separate runs when checkpointing, thus, it's
probably better that they remain method parameters.\\
It would be a good idea that the configuration file is in
YAML\cite{ben09} format, since it's an easy-to-use data serialization
standard with one of the goals as to be human readable, easily
parsable with Ruby.
\subsection{Grid'5000}
As mentioned before, the current framework implementation is fairly
Grid'5000 specific: it uses its API through XPflow to perform tasks
such as node allocation or image deployment. While it's currently
sufficient, in the long run, it would surely be better if the
framework was transformed into a more generic piece of software,
making it possible to use it on other systems.
\subsection{Metadata collection}
Currently, metadata is only written on the standard output. It would
be a useful feature if the framework would save its produced metadata
in a JSON format file and then send it to a permanent location, a
"trace archive". Another possibility would be to use SQL: we could
upload the metadata in a database, created specifically for that
purpose. Then, that database could be queried, for example for
experiments run on specific node(s). A RESTful web service could be
created, providing a user-friendly interface to post queries.
 % Conclusion

%% ----------------------------------------------------------------
\label{Bibliography}
\lhead{\emph{Bibliography}}  % Change the left side page header to "Bibliography"
\bibliographystyle{unsrtnat}  % Use the "unsrtnat" BibTeX style for formatting the Bibliography
\bibliography{Bibliography}  % The references (bibliography) information are stored in the file named "Bibliography.bib"

\end{document}  % The End
%% ----------------------------------------------------------------
