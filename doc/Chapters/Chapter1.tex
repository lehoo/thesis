% Chapter 1

\chapter{Introduction}
\label{Chapter1}
\lhead{Chapter 1. \emph{Introduction}}
Simulation has become increasingly popular in the scientific community
lately because of its many advantages over using real systems in
certain situations. Such an advantage is, for example, that we don't
need to devote lots of money and other resources on acquiring the -
possibly large - system we need for our experiments, or that we can
easily switch between configurations, without having to worry about
how to modify it or how to obtain a new one. SimGrid is a
generic simulation framework, which can be used to simulate multiple
kinds of distributed systems, such as Clouds, Grids or clusters. MPI
is an inter-process communication protocol, one of the most
prominently used in distributed computing currently. SMPI is a
framework that connects SimGrid and MPI: it's part of the SimGrid
project with the goal of running MPI applications in a simulated
environment. SMPI is currently under heavy development. Creating
generic network models is an ongoing effort and it is one of the
development directions that gets the most focus by the development
team. The development process needs continuous validation, which is
done by running real-life MPI benchmarks (using possibly multiple
different implementations), as well as running the same benchmarks
with SMPI. The results can be used to check how accurate the
simulation was. Currently, this testing process consists of multiple
different, manually performed steps. It's tedious, repetitive work,
but as mentioned before, it's necessary. This thesis is about the
implementation of a framework for automating such tests, introducing
as much automation into the process as possible, providing the
possibility to create test workflows that can be run with minimal user
interaction. The goal is to speed up the testing process, making it
possible to acquire proportionally more test results, thus helping the
development team to make progress.
\\
Distributed computing is a very important concept in computer
science. There are multiple types of large-scale distributed
environments that can be used for either production purposes or for
research. Parallelism is also used in a lower level, for example in
graphics processing, where we can have multiple graphics chips in one
computer to do the task.\\
When doing distributed computing, communication between the processes
becomes a very important concern, since it can pose a relatively large
overhead compared to sequential problem-solving - to make parallelism
worthwhile, we have to make sure that the speedup provided by the
distribution of tasks makes up for the communication overhead. To
achieve this, task distribution needs to be carefully planned and the
communication protocols are needed to be optimized. A widely utilized
communication protocol that has been under development for many years
is provided by the MPI\cite{mpif12} inter-process, language-independent
communication API. MPI itself is just a specification, which has
multiple existing implementations. The most widely used
ones include OpenMPI\cite{ompi04} and MPICH\cite{mpich12}.\\
Setting up a distributed environment is a complicated endeavour: it
needs both human and monetary resources. When doing research on
distributed computing, our needs possibly exceed the use of just one
single platform: we would like to test how our experiments fare on
multiple different environments. Such environments are not always at
our disposal. This is why simulation has become a very important field
of research. SimGrid\cite{clq08} is a project providing a wide range
of features in this regard: it is a scientific instrument that can be
used to simulate large-scale distributed systems in order to study
their behavior by evaluating and analyzing the results of parallel
experiments run with the simulator. As mentioned before, inter-process
communication is a very important concern for these experiments. SMPI
is a framework that is part of the SimGrid project. This framework
makes it possible to simulate the execution of parallel applications
that use the MPI standard. This simulation can be done on a single
node.\\
SMPI is a framework that has been validated in the past by experiment
results. Such results have been documented and published, for example
in \cite{csgscq11}. Results are needed to be reproducible, by
providing that the conducted experiments are repeatable. However,
currently, the testing process to procure such results
consists of multiple steps, many of them being manual configuration
steps, such as the allocating of nodes, the creation of a file
containing the allocated nodes or the distribution of the benchmark's
runnable between the nodes. These tedious test processes lack a
universal, user-friendly guide to help other researchers reproduce the
acquired results in order to do further validation. Such a guide would
also help in repurposing, extending our experiments by changing
parameters, switching the underlying configuration, etc. Providing the
possibility to easily repeat the experiments conducted for our paper
is very important: our goal when writing a paper is not only to
announce our results but also to convince our readers that our results
are correct.\cite{m10} The best way to prove that we are right is if
we provide a straightforward way for anyone to repeat our experiments
and to see for themselves that our results stand. Also, if other
scientists are unable tp repeat our experiments, that means that they
are unable to profit from them: to incorporate our results in their
research, which may or may not be related to our field, which is the
main point of scientific collaboration.\\
SMPI is a well-documented and working framework, but also an active
project and as such, it is under constant development. Currently, a
lot of resources are spent on developing generic network models for
the simulator. Extensive testing is needed for the development
process, since continuous validation is necessary to see whether or
not we are heading in the right direction, whether or not the
simulator correctly represents the real-world behavior. This
testing process, as previously mentioned, is currently very
time-consuming. This doesn't only limit the number of tests that can
be run, but also limits the reproducibility of the
results that are achieved with SMPI, the importance of which has been
discussed in the previous paragraph. By constructing a framework that
simplifies the testing process, more reliable and verifiable results
could be produced, as well as it would make the SMPI project members'
lives easier. It is important to include as much automation in the
framework as possible, since other researchers that want to repeat the
experiments might not be computer experts. Our processes can't be
fully automated though. For example, certain parameters, such as the
runnable, or the number of nodes to allocate have to be set - but we
aim to keep such non-automated steps minimal.\\
This thesis discusses how such a framework could be built and provides
an implementation, utilizing the XPflow\cite{bn12_2} experimentation
engine. XPflow is a fairly new project, constantly under development,
but it's fairly stable and suitable for our project. It's designed to
help automating experiments, utilizing a top-down approach taken from
business workflows.
