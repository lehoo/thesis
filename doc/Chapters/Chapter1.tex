% Chapter 1

\chapter{Introduction}
\label{Chapter1}
\lhead{Chapter 1. \emph{Introduction}}
Distributed computing has become a very important subject in computer
science. There are multiple types of large-scale distributed
environments that can be used for either production purposes or for
research. Parallelism is also used in a lower level, for example in
graphics processing, where we can have multiple graphics chips in one
computer to do the task.\\
When doing distributed computing, communication between the processes
becomes a very important concern, since it can pose a relatively large
overhead compared to sequential problem-solving - to make parallelism
worthwhile, we have to make sure that the speedup provided by the
distribution of tasks makes up for the communication overhead. To
achieve this, task distribution needs to be carefully planned and the
communication protocols are needed to be optimized. A widely utilized
communication protocol that has been under development for many years
is provided by the MPI\cite{mpif12} inter-process, language-independent
communication API. MPI itself is just a specification, it has to be
implemented. Many such implementations exist, the most widely used
ones include OpenMPI\cite{ompi04} and MPICH\cite{mpich12}.\\
Setting up a distributed environment is a complicated endeavour: it
needs both human and monetary resources. When doing research on
distributed computing, our needs possibly exceed the use of just one
single platform: we would like to test how our experiments fare on
multiple different environments. Such environments are not always at
our disposal. This is why simulation has become a very important field
of research. SimGrid\cite{clq08} is a project providing a wide range
of features in this regard: it is a scientific instrument that can be
used to simulate large-scale distributed systems in order to study
their behavior by evaluating and analyzing the results of parallel
experiments run with the simulator. As mentioned before, inter-process
communication
is a very important concern for these experiments. SMPI is a framework
that is part of the SimGrid project. This framework makes it possible
to simulate the execution of parallel applications that use the MPI
standard. This simulation can be done on a single node.\\
SMPI is a well-documented and working framework, but also an active
project and as such, under constant development. Extensive testing is
always needed, as it is very important that the behavior of the
application is correctly represented by the simulator. This testing
process is currently very time-consuming. This doesn't only limit the
number of tests, but also limits the reproducibility of the results
that are achieved with SMPI. By constructing a framework that
simplifies the testing process, more reliable and verifiable results
could be produced, as well as it would make the SMPI project members'
lives easier. This thesis discusses how such a framework could be
built and provides an implementation, utilizing \#TODO.
