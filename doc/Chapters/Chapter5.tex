\externaldocument{Chapter4}
\externaldocument{Chapter3}

% Chapter 5

\chapter{Evaluation Plan}
\label{Chapter5}
\lhead{Chapter 5. \emph{Evaluation Plan}}

\label{sec:evaluation_plan}
In this chapter, we go over the process of the evaluation of the
implemented framework, the goal of which is to make
conclusions about whether or not we succeeded in achieving the
requirements we specified earlier. The evaluation will be done by
running a multi-step, fairly typical experiment both manually and with
the framework, so we can easily observe the differences.\\
The experiment process used in the evaluation will consist of the same
steps as mentioned before in the Implementation chapter as an example
MPI experiment process (see \ref{fig:xpflow_example2}), used to collect
RL traces, which can be used for example for the development of
SMPI. The steps in the process are very generic though: we need to be
more precise about certain details of the experiment.
\section{Experiment specification}
\subsection{The benchmark}
The benchmark used for evaluation is taken from the NAS Parallel
Benchmarks suite\cite{jfy99}, which is a small set of benchmarks
designed for performance testing of parallel systems. The benchmark
chosen for our current purpose is called \emph{lu.B.8}. The name
consists of 3 parts. The first part, \emph{lu} indicates what the
experiment is about: as its name suggests, the LU benchmark solves a
system of equations represented with a matrix, with the LU
factorization method.\\
The second part of the name, \emph{B} is an
indication about the complexity of the problem that the benchmark
solves. The NPB suite defines so-called "problem classes" for its
benchmarks. B is in the middle "standard" complexity category, being
more complex than the Small (S), the Workstation-size (W) and A, the
least complex "standard" problem size, but less complex than C and the
larger test problem sizes.\cite{d13}\\
Finally, the number \emph{8} at the end of the benchmark's name
indicates how many parallel processes it is intended for to be solved:
for our purposes, we choose it to be 8, thus, we will allocate 8 nodes
for our experiment, each of them running 1 MPI process to solve the
problem.\\
It's important to note that when doing the experiment, we assume that
the benchmark is already compiled, in a specified place in the user's
home folder: /home/dlehoczky/NPB3.3/NPB3.3-MPI/bin/lu.B.8. For the
user \emph{dlehoczky}, the environment variable \emph{\$NPB\_DIR} is
set to /home/dlehoczky/NPB3.3/NPB3.3-MPI, thus it can be used as a
shortcut when accessing the benchmark. Also, it is assumed that
ssh-keys are set up correctly so the user can reach the specified
Grid'5000 site without any password.
\subsection{The environment}
For our experiment, we use the Grid'5000 testbed, discussed in greater
detail before, in \ref{sec:environment}. As we said before, there are
many different sites to choose from. During development,
many tests were needed to be done. Sometimes one or two sites were
undergoing maintenance or became unstable for various reasons. Towards
the end of the development process, the Lille site proved to be
reliable in terms of availability, this is why it was chosen as the
site to run our final tests on.
\subsection{The OS image}
\label{sec:image}
We use a customized image called
\emph{wheezy-x64-big-lehoo}. This image contains all the necessary
tools required to run the described process. These tools were
mentioned before, at \ref{sec:rl_traces}, when we talked about how to
collect RL traces, which this experiment is about. To reiterate, let's
sum up what our customized image contains in order to execute our
experiment process:
\begin{itemize}
\item OpenMPI 1.6.4;
\item the \emph{TAU}\cite{sm06} profiling tool;
\item the PAPI\cite{mbdh99}\cite{lmmsl01} interface to low-level
hardware counters (configured so it's linked to TAU, which uses it for
tracing);
\item the \emph{Program Database Toolkit (PDT)}\cite{lcmsmrr00} (also
configured to be linked to TAU);
\item the \emph{trace\_gather}\cite{ms11} MPI program;
\item \emph{Akypuera}\cite{s13}, a library to trace mpi applications
and generate paje trace files. (In our experiment, we only use
its \emph{tau2paje} trace converting script.)
\end{itemize}
\subsection{The experiment process}
\label{sec:experiment_process}
As mentioned before, the experiment process will consist of the same
steps as in the example discussed before
(see \ref{fig:xpflow_example2}).\\
To prepare our experiment process, we start by
logging in to the frontend with our user. Then, we start an
interactive job, allocating the desired number (in our case, 8) of
nodes for a specified amount of time. We won't specify what nodes we
want, the system will decide which ones we get. Since there might be
performance disparities between two different sets of nodes, we make
sure to use the exact same set of nodes in the two experiments (the
one done manually and the one done with the framework). We do this by
running the experiment with the framework first, then connecting to
the job started by the framework and repeating the experiment
manually, starting from the deployment part. The node allocation is
not really an important part of the experiment, thus, it's not a
problem that it's omitted from the manual execution.\\
As part of the process that we conduct in both experiments, first, we
deploy the previously mentioned (\ref{sec:image}) customized image on the
nodes. Then we broadcast the runnable (\emph{lu.A.8}) to every node's
/tmp directory. After that comes the step where we disable all but one
core on every node. The reasoning for this step has been discussed
previously (see \ref{sec:multiple_cores}). This concludes the
preparation stage.\\
After that comes the execution of the benchmark. We use OpenMPI 1.6.4,
installed on our image.\\
After the running of the benchmark, comes the post-processing. If we
compiled our benchmark correctly with TAU, one trace file (.trc)
and one event file (.edf) is generated for each MPI process. The
traces are generated on the node of execution. This is why first,
we run the \emph{trace\_gather}\cite{ms11} script to collect the traces
and event files scattered across all the allocated nodes to the head
node. When we have all the files on our head node, we use
TAU's \emph{tau\_treemerge.pl}, which is a script that merges all our
trace files and event files into one trace and event file
respectively, also trying to account for any clock skew
(see \ref{sec:clock_synch}) between the trace files with
post-processing methods. Finally, when we have one merged trace and
one event file, we can convert our TAU trace to a format that is
compatible with the Pajé\cite{cob00} visualization tool. Visualizing
the traces makes it easier to analyze and compare. For the conversion,
we use Akypuera's\cite{s13} \emph{tau2paje} script.\\[0.5cm]
In this chapter, we went over the specifics of the evaluation of our
implementation. First, we talked about the chosen benchmark to gather
traces from, which is called \emph{lu.B.8}. We alse explained there
what the name stands for: LU indicates the problem the benchmark
solves, B is the problem class and 8 indicates how many nodes the
benchmark is compiled for. We also discussed what the chosen
environment is going to be (Grid'5000) and what operating system image
we are going to deploy on our nodes before running the benchmarks on
them. Finally, we went over the experiment process: the RL trace
collection for a given benchmark. We mentioned that our experiment
process will be done both with and without the framework to
demonstrate how the two methods differ from each other. We discussed
the preparation steps, the running of the benchmark and the
post-processing part. Now, in the next chapter, we'll take a look at
the results of our experiments.
